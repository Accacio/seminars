\documentclass[aspectratio=169]{audition-beamer}

\renewcommand{\footnoterule}{}

\usepackage[francais]{babel}
\input{preamble}

\usetikzlibrary{positioning,calc}
\usetikzlibrary{arrows.meta}
\usetikzlibrary{decorations.pathmorphing}

% Adapted from https://tex.stackexchange.com/a/396754/28146

% \setbeameroption{show notes on second screen}
\draft
% \webcast
% \setbeamercolor{alerted text}{fg=supelecRed!20!red!80}
% \usetikzlibrary{overlay-beamer-styles}
% \includeonlyframes{current,current1,current2,current3,current4,current5,current6,current7}

\bibliography{bibliography.bib}

\title[Discuter] % (optional, use only with long paper titles)
{Discuter:}

\subtitle
{Ontologies + NLP + Minecraft = \raisebox{-.5em}{\includegraphics[width=1.5em]{thinking.png}}}

\author[Rafael Accácio Nogueira] % (optional, use only with lots of authors)
{Rafael Accácio NOGUEIRA\\
  \texttt{rafael.accacio.nogueira@gmail.com}}

% \institute[IETR --- CentraleSupélec] % (optional, but mostly needed)
% {
% }


\day29 \month08 \year2024
\makeatletter
\date{\@date\ @ Toulouse}
\makeatother
% {\textbf{Audition MCF 4101 0067 \\École Centrale de Lyon / Laboratoire Ampère}
%   \\
%   30 mai 2023 @ Écully\\
%   % \begin{minipage}{.3\textwidth}
%   %   \centering
%   %   %   \includegraphics[width=2cm]{logos/IETR_2022.png}
%   % \end{minipage}
%   % \hfill
%   \begin{minipage}{\textwidth}
%     \centering
%     \vspace{10pt}
%     \includegraphics[width=1.5cm]{qrPresentation.png}
%     % qrencode https://gitlab.com/Accacio/audition_mcf_2023/-/raw/main/presentation.pdf -o ~/git/SysTol21/img/qrPresentation.png
%     % \href{https://bit.ly/3g3S6X4}{https://bit.ly/3g3S6X4}
%   \end{minipage}
%   % \begin{minipage}{.3\textwidth}
%   %   \centering
%   %   \vspace{10pt}
%   %   %   \includegraphics[width=2cm]{logos/supelec.jpeg}
%   % \end{minipage}
% }
% % - Either use conference name or its abbreviation.
% % - Not really informative to the audience, more for people (including
% % yourself) who are reading the slides online

% \subject{}

% % \logo{\includegraphics[width=1.5cm]{logos/supelec.jpeg}}

% % Delete this, if you do not want the table of contents to pop up at
% % the beginning of each subsection:
\AtBeginSection[]
{
  \begin{frame}<beamer>{Sommaire}
    \tableofcontents[sectionstyle=show/hide,subsectionstyle=show/show/hide]
  \end{frame}
}

\begin{document}

\begin{frame}[plain]
  \titlepage%
\end{frame}

% \begin{frame}[plain]
%   \titlepage%
%   \note{45 minutes !!!!\\}
%   \script{Good afternoon, thank you all for being here.}
%   \script{I'm Rafael Accácio and I'm going to present my work on the security of distributed model predictive control under false data injection.}
% \end{frame}


\begin{frame}{DISCUTER}

  
\resizebox{\textwidth}{!}{\alert{D}ialogue \alert{I}nteractif \alert{S}tructuré, \alert{C}onsolidé et \alert{U}nifié pour la réalisation de \alert{T}âches \alert{E}n \alert{R}obotique}\pause
\\~\\
Étudier la collaboration entre agents pour des tâches complexes en utilisant:
\begin{itemize}[<+(1)->]
  \item langage naturel
  \item événements non-linguistiques
  \item contenu des bases de connaissances
  \item états cognitifs modélisés
\end{itemize}
\pause
~\\
\centering
Mais quelle activité/outils pour tester?
\end{frame}

\begin{frame}{Collaborative Dialogue}
{\small \cite{Narayan-ChenEtAl2019}}
\centering
\onslide<2->{\scalebox{0.7}{\begin{tikzpicture}
\node[inner sep=0cm] (image) {\includegraphics[width=.7\linewidth]{Narayan-ChenEtAl2019.png}};
\only<-3>{\node[fill,white,rectangle,anchor=east,minimum width=.352\linewidth,minimum height=5.55cm] at (image.east) {};}
\only<-2>{\node[fill,white,rectangle,anchor=south west,minimum width=.35\linewidth,minimum height=3cm] at (image.south west) {};}
\node[anchor=north] at (image.south) {\tiny Source:  \citeauthor{Narayan-ChenEtAl2019} \citeyear{Narayan-ChenEtAl2019}};
\end{tikzpicture}}}

\addtocounter{beamerpauses}{3}
\pause
Collection des Dialogues\footnote<.(1)->{Minecraft Dialogue Corpus \raisebox{-.5ex}{\includegraphics[width=1em]{github-mark.png}}} \pause\textrightarrow{} Génération d'énontiations\pause\ (Architect est un robot)

\pause
~\\
Et si le Builder était un robot?\pause
\\
\small \citeauthor{JayannavarEtAl2020}~\citeyear{JayannavarEtAl2020} (Pas de raisonnement spatial)
\end{frame}

\begin{frame}{Travail}

  \vfill
  Établir l'architecture pour créer une preuve de concept\pause\ qui associe
  \begin{itemize}
    \item Traitement automatique des langues (NLP)
    \item Overworld/Ontologenius (Raisonnement spatial et bases de connaissances)
  \end{itemize}


  % \tableofcontents
  \onslide<+(1)->{\tableofcontents[subsectionstyle=hide/hide/hide]}

\end{frame}

\section{Reliant Minecraft/Overworld/NLP}

\subsection{Architecture}
\begin{frame}{Architecture}

\centering
\vfill

\resizebox{0.6\linewidth}{!}{
\begin{tikzpicture}[node distance={.1cm and 1cm}]
  % MINECRAFT
  \node[image] (architect) at (-1,0) {\includegraphics[width=2cm]{architect.png}};
  \node[below=of architect] (architect_text) {\small Architect};
  \node[image,right=of architect] (builder) {\includegraphics[width=2cm]{builder.png}};
  \node[below=of builder] (builder_text) {\small Builder};
  \onslide<2->{\node[below=of architect_text] (minecraft_text) {\Large Malmö\footnote<2->{\raisebox{-.5ex}{\includegraphics[width=1em]{microsoft.png}} Malmö 2016  (API pour l'apprentissage par renforcement) \raisebox{-.5ex}{\includegraphics[width=1em]{github-mark.png}}}};}
  \node[inner sep=0cm,right=-.1cm of minecraft_text] {\includegraphics[clip,width=2cm,trim={1cm 30cm 0 25cm}]{minecraft_logo.png}};

  \coordinate (malmodouble) at ($(architect.west |- minecraft_text.south west) + (builder.north east)$);
  \coordinate (malmo) at ($0.5*(malmodouble)$);

  \node[draw,rectangle,minimum width=6.cm,minimum height=3.5cm] (malmo_node) at (malmo) {};


\onslide<3->{
  \node[draw,rectangle,fill=LemonChiffon1,above=4.7cm of malmo,minimum width=5cm,minimum height = 2cm,align=center] (malmo_interface) {\Large Malmö Interface\\ \small nœud ROS};
}
\onslide<2->{
  \draw[<-]  ([xshift=-.5cm]malmo_interface.south-|builder.north) -- ([xshift=-.5cm]builder.north) node[above,sloped,midway] {Stats};
  \draw[->] ([xshift=0cm]malmo_interface.south-|builder.north) -- ([xshift=.0cm]builder.north) node[above,sloped,midway] {Commands};
  \draw[->] ([xshift=.5cm]malmo_interface.south-|builder.north) -- ([xshift=.5cm]builder.north) node[above,sloped,midway] {Chat};

  \draw[<-] ([xshift=.25cm]malmo_interface.south-|architect.north) -- ([xshift=.25cm]architect.north) node[above,sloped,midway] {Chat};
  \draw[<-] ([xshift=-.25cm]malmo_interface.south-|architect.north) -- ([xshift=-.25cm]architect.north) node[above,sloped,midway] {Stats};
}

\onslide<5->{
  \node[draw,rectangle,fill=LightBlue1,above left=2.cm of malmo_interface,minimum width=4cm,minimum height= 2cm,align=center] (dialog_manager) {\Large Dialog Manager\\ \small nœud ROS};
}

\onslide<4->{
  \draw[->] (malmo_interface.190) -| (dialog_manager.250) node[above,sloped,near start] {Chat from Architect};
}
\onslide<6->{
  \draw[<-] (malmo_interface.170) -| (dialog_manager.290) node[above,sloped,near start] {Chat to Builder};
}

\onslide<8->{
  \node[draw,rectangle,fill=Tan1,above right=2.cm of malmo_interface,minimum width=4cm,minimum height= 2cm,align=center] (reasoner) {\Large Planner\\ \small nœud ROS};
}

\onslide<10->{
  \draw[<-] (malmo_interface.10) -| (reasoner.230) node[above,sloped,near start] {Actions};
}
\onslide<9->{
  \draw[->] (malmo_interface.40) |- (reasoner.205) node[above,sloped,near end] {World State};
}
\onslide<7->{
  \draw[->] ([shift={(0cm,.8cm)}]dialog_manager.east) -- ([shift={(0cm,.8cm)}]reasoner.west) node[above,sloped,near end] {Filtered Tasks};}
\onslide<10->{
  \draw[<-] ([shift={(0cm,0.2cm)}]dialog_manager.east) -- ([shift={(0cm,0.2cm)}]reasoner.west) node[above,yshift=-0.2em,sloped,near end] {World Updates};}
\onslide<11->{
  \draw[<-] ([shift={(0cm,-0.4cm)}]dialog_manager.east) -- ([shift={(0cm,-0.4cm)}]reasoner.west) node[above,sloped,near end] {Common Errors};}
\end{tikzpicture}
}
\end{frame}

\subsection{Flux de données}

\begin{frame}{Flux de données}

  \vfill
  \vfill
  \begin{itemize}
    \item Malmö API \only<2->{(\texttt{json})}
          \begin{itemize}[<.(2)->]
            \item Stats
            \item Chat
            \item Grid Observations
            \item Commands
          \end{itemize}
    \item ROS
          \begin{itemize}[<.(3)->]
            \item Chat (\texttt{StampedString})
            \item (Filtered) Tasks (\texttt{StampedString})
            \item Common Errors (\texttt{StampedString})
            \item World Updates (\texttt{StampedString})
            \item Positions agents/blocs \textrightarrow\ Overworld (\texttt{*Poses})
          \end{itemize}
    \item Ontologenius
          \begin{itemize}[<.(4)->]
            \item Positions des agents/blocs \tikzmark{a}
            \item Couleurs des blocs \tikzmark{ab}
          \end{itemize}
  \end{itemize}
      \onslide<4->{\begin{tikzpicture}[overlay, remember picture,decoration={brace,amplitude=2pt}]
      \draw[decorate,thick] (a.north east) -- (a.north east |- ab.south east) node[midway, right=0.1cm,text=black,text width = 2in] {Inheritance\\ ObjectProperty};
    \end{tikzpicture}
  }

\end{frame}

\subsection{Tâches}
\begin{frame}{Tâches}
\scalebox{.35}{
\begin{tikzpicture}[node distance={.1cm and 1cm},remember picture,overlay,yshift=-8cm,xshift=33.5cm]
  % MINECRAFT
  \node[image] (architect) at (-1,0) {\includegraphics[width=2cm]{architect.png}};
  \node[below=of architect] (architect_text) {\small Architect};
  \node[image,right=of architect] (builder) {\includegraphics[width=2cm]{builder.png}};
  \node[below=of builder] (builder_text) {\small Builder};
  \node[below=of architect_text] (minecraft_text) {\Large Malmö};
  \node[inner sep=0cm,right=-.1cm of minecraft_text] {\includegraphics[clip,width=2cm,trim={1cm 30cm 0 25cm}]{minecraft_logo.png}};

  \coordinate (malmodouble) at ($(architect.west |- minecraft_text.south west) + (builder.north east)$);
  \coordinate (malmo) at ($0.5*(malmodouble)$);

  \node[draw,rectangle,minimum width=6.cm,minimum height=3.5cm] (malmo_node) at (malmo) {};

  \node[draw,rectangle,fill=LemonChiffon1,above=4.7cm of malmo,minimum width=5cm,minimum height = 2cm,align=center] (malmo_interface) {\Large Malmö Interface\\ \small nœud ROS};

  \draw[<-]  ([xshift=-.5cm]malmo_interface.south-|builder.north) -- ([xshift=-.5cm]builder.north) node[above,sloped,midway] {Stats};
  \draw[->,alt=<5>{red,ultra thick}] ([xshift=0cm]malmo_interface.south-|builder.north) -- ([xshift=.0cm]builder.north) node[above,sloped,midway] {Commands};
  \draw[->] ([xshift=.5cm]malmo_interface.south-|builder.north) -- ([xshift=.5cm]builder.north) node[above,sloped,midway] {Chat};

  \draw[<-,alt=<2>{red,ultra thick}] ([xshift=.25cm]malmo_interface.south-|architect.north) -- ([xshift=.25cm]architect.north) node[above,sloped,midway] {Chat};
  \draw[<-] ([xshift=-.25cm]malmo_interface.south-|architect.north) -- ([xshift=-.25cm]architect.north) node[above,sloped,midway] {Stats};


  \node[draw,rectangle,fill=LightBlue1,above left=2.cm of malmo_interface,minimum width=4cm,minimum height= 2cm,align=center] (dialog_manager) {\Large Dialog Manager\\ \small nœud ROS};


  \draw[->,alt=<2>{red,ultra thick}] (malmo_interface.190) -| (dialog_manager.250) node[above,sloped,near start] {Chat from Architect};

  \draw[<-] (malmo_interface.170) -| (dialog_manager.290) node[above,sloped,near start] {Chat to Builder};


  \node[draw,rectangle,fill=Tan1,above right=2.cm of malmo_interface,minimum width=4cm,minimum height= 2cm,align=center] (reasoner) {\Large Planner\\ \small nœud ROS};


  \draw[<-,alt=<4>{red,ultra thick}] (malmo_interface.10) -| (reasoner.230) node[above,sloped,near start] {Actions};

  \draw[->] (malmo_interface.40) |- (reasoner.205) node[above,sloped,near end] {World State};


  \draw[->,alt=<3>{red,ultra thick}] ([shift={(0cm,.8cm)}]dialog_manager.east) -- ([shift={(0cm,.8cm)}]reasoner.west) node[above,sloped,near end] {Filtered Tasks};
  \draw[<-] ([shift={(0cm,0.2cm)}]dialog_manager.east) -- ([shift={(0cm,0.2cm)}]reasoner.west) node[above,yshift=-0.2em,sloped,near end] {World Updates};
  \draw[<-] ([shift={(0cm,-0.4cm)}]dialog_manager.east) -- ([shift={(0cm,-0.4cm)}]reasoner.west) node[above,sloped,near end] {Common Errors};
\end{tikzpicture}
}


  \vfill
  \pause
  Architect: ``Put a blue block on your left''

  \vfill
  \small
  \begin{description}[<+(1)->]
    \item[Filtered Task] possible tâche complexe (\texttt{PUT(left,blue)})
    \item[\only<.(3)->{\alert{\textrightarrow} \stepcounter{beamerpauses}}Task] liste de tâches (\texttt{TURN(LEFT)}, \texttt{PUT(blue)}, \texttt{TURN(RIGHT)})
    \item[Commands] Action faite par Builder (\texttt{turn}, \texttt{setPitch}, \texttt{place})
  \end{description}

  \vfill
  \pause
  Les tâches doivent
  \begin{itemize}[<+(1)->]
    \item être réactives (c-à-d, elles dépendent de l'état)
    \item avoir liste de pré-conditions
    \item avoir liste de post-conditions
  \end{itemize}

\end{frame}

\subsection{Planificateur}
\begin{frame}{Planificateur}{Partielment implementé }

  \pause
  Planificateur doit :
  \begin{itemize}[<+(1)->]
    \item[\only<9->{\checkmark}] avoir une bibliothèque de tâches
    \item[\only<9->{-- }] avoir accès au raisonnement
    \item[] avoir une bibliothèque de conditions et tâches associées (pré et post tâches)
    \item[] tester si action est faisable
    \item[\only<9->{\checkmark}] envoyer des m-à-j des actions au Dialog Manager
    \item[\only<9->{\checkmark}] envoyer erreur au Dialog Manager sinon
  \end{itemize}
\end{frame}


\section{Tâches réalisées}
\begin{frame}{Tâches réalisées}
  \pause
  \vfill
Point de départ: Stage Ismail (Enregistrement d'états en \texttt{.json})
  \vfill
\begin{itemize}[<+(1)->]
  \item MàJ Malmö pour fonctionner sur Ubuntu 20.04
  \item Image docker
  \item Reproduction scénario de collecte (Malmö pur)
  \item Création de l'interface Malmö/ROS/Overworld
  \item Mock-up du «Dialog Manager» \tikzmark{a}
  \item Mock-up du Planner \tikzmark{ab}
\end{itemize}

\onslide<9->{\begin{tikzpicture}[overlay, remember picture,decoration={brace,amplitude=2pt}]
    \draw[decorate,thick] (a.north east) -- (a.north east |- ab.south east) node[midway, right=0.1cm,text=black,text width = 2in] {regex FTW!};
    \end{tikzpicture}
  }
\end{frame}

\section{Petite démo}

\section{Conclusion}

\begin{frame}{Conclusion}
  \begin{itemize}[<+->]
    \item Preuve de concept
          \begin{itemize}[<+->]
            \item Interaction avec Minecraft utilisant ROS
            \item Intégration partiel des ontologies
            \item Tâches extensibles par autres tâches
            \item Plateforme pour expérimentation avec dialogue et ontologies
          \end{itemize}
  \end{itemize}

  \begin{itemize}[<+->]
    \item Dans l'horizon
          \begin{itemize}[<+->]
            \item Parser plus intelligent avec LLM
            \item ajouter réactivité aux changements d'état
            \item Planification des mouvements (point d'approche etc)
          \end{itemize}
  \end{itemize}
\end{frame}

\begin{frame}[plain]
  \centering
  \vfill
  \begin{minipage}[t]{.5\linewidth}
    \small
    \centering
    Contact\\
% qrencode mailto:rafael.accacio.nogueira@gmail.com?subject=Seminaire RIS Minecraft -o qrContact.png
    \href{mailto:rafael.accacio.nogueira@gmail.com?subject=Seminaire RIS Minecraft}{rafael.accacio.nogueira@gmail.com}

    \vspace{10pt}
    \qrcode[height=2cm]{mailto:rafael.accacio.nogueira@gmail.com?subject=Seminaire RIS Minecraft}
  \end{minipage}
  \ifwebcast{\tikz{\draw[fill=pink,draw=pink] (1.5,0) circle [radius=1.5cm]}}%
  \fi
\end{frame}

\appendix

\end{document}
