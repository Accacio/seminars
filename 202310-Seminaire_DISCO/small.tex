\documentclass[aspectratio=169]{neocampus}
\usepackage[style=authortitle]{biblatex}
\addbibresource{../../bibliography.bib}

\title{Localisation relative \\ de robots mobiles}
\author{Rafael Accácio NOGUEIRA\\
Soheib FERGANI\\ \texttt{\{ranogueira, sfergani\} \\at laas.fr}}

\def\day{7}
\def\month{10}
\def\year{2023}

\newcommand{\eq}[2]{\mbox{$#1=#2$}}
\newcommand{\N}{\mathbb{N}}
\newcommand{\Z}{\mathbb{Z}}
\newcommand{\Zpos}{\mathbb{Z}_{+}}
\newcommand{\Q}{\mathbb{Q}}
\newcommand{\R}{\mathbb{R}}
\newcommand{\C}{\mathbb{C}}
\newcommand{\Np}{N_{\text{p}}}
\newcommand{\T}{^{\mathrm{T}}}
\newcommand{\1}{\mathbf{1}}
\newcommand{\0}{\mathbf{0}}
\newcommand{\abs}[1]{\left\lvert#1\right\rvert}
\newcommand{\norm}[1]{\left\lVert#1\right\rVert}


\newcommand{\vectorize}[1]{\mathrm{vec} (#1)}
\newcommand{\Varepsilon}{\mathcal{E}}
\newcommand{\diff}{\mathop{}\mathopen{}\mathrm{d}}
\newcommand{\set}[1]{\mathcal{#1}}
\newcommand{\graph}[1]{\mathscr{#1}}
\renewcommand{\vec}[1]{\boldsymbol{#1}}
\newcommand{\random}[1]{\underline{#1}}
\newcommand{\randomvec}[1]{{\underline{\vec{#1}}}}
\newcommand{\probability}[1]{\mathbb{P}(#1)}
\newcommand{\pdf}[1]{p(#1)}
\newcommand{\expectation}[1]{\mathbb{E}\left[#1\right]}
\newcommand{\expectationof}[2]{\mathbb{E}_{#1}\left[#2\right]}
\newcommand{\indicator}[1]{\mathbb{1}_{\{#1\}}}
\newcommand{\vecangle}[2]{\langle_{#1}^{#2}}
\newcommand{\until}{\mathbin{:}}
\newcommand{\pseudoinv}[1]{{#1}^{\dagger}}


\newcommand{\setbuild}[2]{\{#1\mid#2\}}
\newcommand{\seq}[2]{\lbrace #2_{0},\ldots,\,#2_{#1} \rbrace}
\newcommand{\hadamard}[2]{#1\circ #2}
\newcommand{\kron}[2]{#1\otimes#2}
\newcommand{\symmetric}{\mathbb{S}}
\newcommand{\semidefpos}{\mathbb{S}_{+}}
\newcommand{\defpos}{\mathbb{S}_{++}}
\newcommand{\elem}[2]{{#2}_{(#1)}}
\renewcommand{\implies}{\Rightarrow}
\renewcommand{\iff}{\Leftrightarrow}
\newcommand{\argmax}{\mathop{\arg\!\max}}
\newcommand{\argmin}{\mathop{\arg\!\min}}
\newcommand{\maximize}{\mathop{\textrm{maximize}}}
\newcommand{\minimize}{\mathop{\textrm{minimize}}}
\newcommand{\minimiser}{\mathop{\textrm{minimiser}}}
\newcommand{\maximiser}{\mathop{\textrm{maximiser}}}


\newcommand{\elemend}{\mathop{\textrm{end}}}
\newcommand{\diag}{\mathop{\textrm{diag}}}
\newcommand{\fix}{\mathop{\textrm{fix}}}
\newcommand{\Proj}[1]{{\textstyle\mathop{\textrm{Proj}}^{#1}}}
\newcommand{\dom}{\mathop{\textrm{dom}}}
\newcommand{\card}{\mathop{\textrm{\#}}}


\usetikzlibrary{positioning,hobby,calc,arrows,overlay-beamer-styles}
\usetikzlibrary{spy,shadows,shapes.symbols,shapes.geometric}
\usetikzlibrary{decorations.pathreplacing,intersections}
\tikzset{%
  show curve controls/.style={
    postaction={
      decoration={
        show path construction,
        curveto code={
        %   \draw [blue]
        %     (\tikzinputsegmentfirst) -- (\tikzinputsegmentsupporta)
        %     (\tikzinputsegmentlast) -- (\tikzinputsegmentsupportb);
          \fill [red, opacity=0.5]
            (\tikzinputsegmentsupporta) circle [radius=.5ex]
            (\tikzinputsegmentsupportb) circle [radius=.5ex];
        }
      },
      decorate
}}}

\begin{document}

\begin{frame}{Approximation par ensembles connus}
  \centering
  \begin{overlayarea}{\textwidth}{.25\paperheight}
    \begin{itemize}
      \item<+(1)-> Compromis entre complexité de calcul et conservatisme
            \begin{itemize}
              \item<+(1)-> Rapide mais conservative: {\color<3>{green!50!white}{intervalles}}, {\color<4>{blue!50!white}{ellipsoïdes}}, {\color<5>{olive!50!white}{zonotopes}}
              \item<+(3)-> Lent mais serré (normalement hors-ligne): \alert<6>{sous-pavage (SIVIA)}, {\color<7>{neocampus_yellow}{polytopes}}
            \end{itemize}
      \item<+(4)-> Zonotope Contraint\only<.(4)->{\footnote{\cite{ScottEtAl2016} (\citeyear{ScottEtAl2016})} }(une autre representation de polytopes)
    \end{itemize}
  \end{overlayarea}

  \vfill
  \begin{overlayarea}{\textwidth}{.3\paperheight}
    \begin{minipage}[c]{.45\linewidth}
      \scalebox{3}{
        \begin{tikzpicture}
          % \draw [help lines] (-1,-1) grid [step=.1cm] (1,1);
          \node[] (car_purple_position) {};

          % interval
          \onslide<3->{
            \draw[green!50!white,fill=green!50!white,alt=<4->{opacity=0.2}] (-0.65,-0.33) rectangle ++(1.15,.73);
          }

          % ellipse
          \onslide<4->{
            \draw[rotate=20,blue!50!white,fill=blue!50!white,alt=<5->{opacity=0.2}] ($(car_purple_position)+(-0.075,0)$) ellipse (.7cm and .44cm);
          }

          % zonotope
          \onslide<5->{
            \def\s{0.42}
            \def\a{0.88}
            \draw[olive!50!white,fill=olive!50!white,xshift=-0.71cm,yshift=0.04cm,alt=<6->{opacity=0.2}] (0,0) -- ++(60:\s) -- ++(0:\a) --++(-60:\s) -- ++(240:\s) --++(180:\a) -- ++(120:\s);
          }

          % constrained zonotope/ polytope
          \onslide<7->{
            \draw[neocampus_yellow,fill=neocampus_yellow] (-0.65,-0.1) -- ++(90:0.2) -- ++(45:0.42) --++(0:0.4) -- ++(-30:0.53) -- ++(-90:0.32) -- ++(210:0.28) --++(180:0.45) --++(160:0.49) -- cycle;
          }

          \path[draw,use Hobby shortcut,closed=true,violet!50!white,fill=violet!50!white]
          ($(car_purple_position)+(.5,0)$) .. ++(-.3,.3) .. ++(-.1,.05) .. ++(-0.6,-0.1) ..++(-0.1,-0.4) ..++(0.2,-0.1);
          \node[] (car_purple) at (car_purple_position) {\includegraphics[height=1cm,angle=90]{car_purple}};

        \end{tikzpicture}
      }
    \end{minipage}
    \hfill
    \begin{minipage}[c]{.45\linewidth}
      \visible<6->{
        \includegraphics[trim={100 20 50 4},clip,width=.6\textwidth]{sivia}
      }
    \end{minipage}
  \end{overlayarea}
\end{frame}
\end{document}
